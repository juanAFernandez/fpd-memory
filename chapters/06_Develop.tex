\chapter{Develop}


\section{Api Gateway}
\subsection{Cloud Endpoint spike}

API Rest with flask, not Google CloudEndpoints

Before to spends some weeks with the recomended technology by Google, was
detected some drawbacks that was upper than the benefits of this technology,
in spite of that it had a better performance.

This is an example of the spike related, it can see in the first phases of the
project.

\begin{lstlisting}[language=python,frame=none]
...

import endpoints
from protorpc import messages

...

class Alumno(messages.Message):
  nombre = messages.StringField(1)
  apellidos = messages.StringField(2)
  id = messages.StringField(3)

class ListaAlumnos(messages.Message):
  alumnos = messages.MessageField(Alumno, 1, repeated=True)

...

@endpoints.api(name='gateway', version='v1')
class GatewayApi(remote.Service):
  """Gateway API v1."""

  @endpoints.method(message_types.VoidMessage, ListaAlumnos,
                    #path=nombre del recurso a llamar
                    path='alumnos/getAlumnos', http_method='GET',
                    #Puede que sea la forma en la que se llama desde la api:
                    #response = service.alumnos().listGreeting().execute()
                    name='alumnos.getAlumnos')

  def getAlumnos(self, unused_request):
    ...
    students_list = []
    ...
    # Logic to get the students list from another service with maybe more logic.
    ...
    alumnosItems.append(Alumno( id=idAlumno, nombre=nombreAlumno.decode('utf-8'), apellidos=apellidosAlumno.decode('utf-8') ) )
    return ListaAlumnos(alumnos=alumnosItems)
...
\end{lstlisting}
About 2000 lines in the endpoint first aproximation of the Gateway API microService




Now take a look to the \emph{flask} version:

\begin{lstlisting}[language=python,frame=none]
@api_gateway.route('/students', methods=['GET'])
def get_students():
    ...
    students_list = []
    ...
    # Logic to get the students list from another service with maybe more logic.
    ...
    return students_list
\end{lstlisting}

\section{Api Gateway}

\section{Teaching Data Base microService}

\subsubsection{Optional subjects, the ``class'' Table.}

In the domain of the problem can be exists optative subjects and is
needed search a way to implement this because has a specific details
that aren't like the rest.

\paragraph{The details.}

The studies plan forces in certain courses to select one or several
optional subjects. For example, if a student has enrollment in 2ºESO
(independently of the group, A, B...) the law and consequently the
studies plan force to the student to choose between some optional
subjects. So, maybe this subjects exists only in this optional case
as `` rare subjects'' but in other cases this are only normal subjects
but that in this course are offer like optional.

A simple example of this is French subject, it in some courses like
3ESO and 4ESO is obligatory but it in Bachiller (the upper level)
is optional because the students can be select if they want make the
final exam with this second language or select another like English
or Greek or Latin p.e.

To obtain this we decide develop a simple solution without change
the original database logic schema. So, how we have an entity that
save the classes and it have three attributes, course, word and level
mainly we going to add three more to this special cases, optative,
groupNumber and subgroupNumber. Like this special cases haven't word
param when they have value word don't have and when the item have
word (A, B, C...) then don't have this special attributes.

Maybe this don't be the best solution, but is a simple in the point
of develop.

Obviously like we can't have two autoincrement values in the same
table definition in MySQL we will need control this programatically,
but is something that we can assume to get our goal easily.

Problema:

La misma ventaja que nos ofrece UNIQUE para el caso de los eliminados
ahora nos muerde aquí. Mientras que allí beneficiaba porque esta clausula
no incluye a los item que tengan campos a null y permite que no de
conflicto en este caso si insertamos un grupo optativo obviamente
deberá tener el camo word a null y si es así podemos tener exactamente
dos grupos iguales sinque de conflicto.

course

1 <null> ESO 1 1 1 0

1 <null> ESO 1 1 1 0

Sin dar conflicto, lo que no puede ser.



está a null o no para le resto de los procedimientos.

Por esta razon decidimos usar el mismo campo word, con una nomenclatura
especial, ya que no se va a ser usada para los grupos generales que
indique que se trata de un grupo optativo y además especifique el
grupo y el subgrupo, en concreto:

OPT\_n\_m

donde n será el número del grupo (que habrá que incrementar prograticamente
a mano) y m el número del subgrupo que tambi\'{n}e habrá que incrementar
a mano en caso de que se creen más.

Esto siempre a falta de un mejor solucion que se adopte en iteraciones
posteriores.

Y en la api nosotros siempre preferimos explícito a implicito.

Requisitos:
\begin{enumerate}
\item Ya que todo se controla desde mysql pero no existe forma de hacerlo
automatico vamos a crear unos disparadores para la inserccion y la
eliminacion para mantener la consistencia de los datos. Cuando se
introduce un nuevo grupo optativo se debe comprobar que existe el
anterior tanto en grupo como en subgrupo para que no exista el subgrupo
4 sin el 3 o el dos. De la misma manera no se podrán eliminar un grupo
4 si el 2 aún existe.
\end{enumerate}

\paragraph{API definition}

@app.route('/entities/<string:kind>', methods={[}'POST'{]})

Para realizar la introduccion de un estudiante al sistema usamos

el re.

No podemos introducir un usuario que no tenga datos, al menos

deberá pasarse el dato nombre que es úniquo mínimo que debe

existir para el tipo alumno.

Cuando se pide una lista de elementos y la peticin se realiza correctamente
pero no se devuelve nada porque no existen items de ese recurso se
envía un 204 Success without content.

Importante: Muchos de los detalles de control de consistencia no pueden
relegarse a la UI, ya que aunque en ella no se deben permitir ciertas
acciones por la logica del sistema, la api no debe permitir su realizacin
aunque se haga de forma manual sin los conectores de la UI.

Por eso no solo la seguridad sino el control de la consistencia debe

estar presente de la mejor forma posible en todas las acciones que
contra esta un usuario puede hacer.

\paragraph{User histories}

El microservicio de Base de Datos acota el dominio de la gestin docente
dentro del sistema, quizs se le cambie el nombre a TmS teachingMicroService.

A través de este microservicio podemos realizar todas las acciones
relativas a.

Este microservicio se basa en una base de datos.

I like

\subsection{Deleted items strategy}

\section{Students Control microService}

\section{Analysis microService}




PONER EJEMPLOS DE COMO LAS ESPECIFICACIONES DE RUTAS EN LA PARTE DE DISEÑO
SE CORRESPONDEN CON LA IMPLEMENTAIÓN EN FLASK
