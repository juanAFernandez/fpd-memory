\chapter{Develop}


\section{Api Gateway microService}

As was discussed above, the principal goal of this service is to offer the
abstraction of the whole system, so it will not be very complex because all
functionality is actually implemented when this service is built.
\intro
Their unique task is to receive the requests, to construe this and do the call
to the service that must answer it and return the data,
maybe modified.
In common case and in the point of the develop are the service only
reply the queries and return the responses without more logic interaction, but
in the future, this is not only that this will do, because is the perfect place
to implement authorization and authentication with ACLs\footnote{ACL of Access Control List
specifies which users or system processes are granted access to objects, as well
as what operations are allowed on given objects} exploiting the fact that through
this service all calls go.
\intro
Another situation which this service would transform the response of a service or
compound a response with the response of few of them is when must be offered a
resource that is the composition of data arising from several services. For example,
when a user profile info is required, the call will be redirected to Teaching Data
Base mService that will return a simple data block. So, this data block does not
have the image of the user, because this is stored in another different service.
Now the role of gateway take sense because will be it who will retrieve the image
from other service, will insert it in the data block (by some way) and will return
the data block complete.
\intro
It also provides some possibilities to have a dynamic and not locking develop process.
How? Easy, because at first the images can be extracted in execution time from some
library that put example images (or something like this) without locking the APIGmS
and also UImS and so, this service of data storage could be developed after, in another
phase of priorities. So, one more time, the architecture help us to develop team domain
based without any blocking.


\subsection{Cloud Endpoint spike}

As we are working with GAE and GCE the first approach is use
their technology, and if in any resource that you can read about this
they are talking about Google Cloud Enpoints must be some good reason.
This technology is based of an own version of RPC protocol developed
by Google and used in their own intern architecture, called gRPC.
\intro
Basically is another implementation of a Remote Procedure Call
system, but according to Google, really fast and nice, and is true, but has some
drawbacks as for example that can you select other common technologies, as in
our case.
\intro
This is an example of the spike related, it can see in the first phases of the
project.

\begin{lstlisting}[language=python,frame=none]
...

import endpoints
from protorpc import messages

...

class Alumno(messages.Message):
  nombre = messages.StringField(1)
  apellidos = messages.StringField(2)
  id = messages.StringField(3)

class ListaAlumnos(messages.Message):
  alumnos = messages.MessageField(Alumno, 1, repeated=True)

...

@endpoints.api(name='gateway', version='v1')
class GatewayApi(remote.Service):
  """Gateway API v1."""

  @endpoints.method(message_types.VoidMessage, ListaAlumnos,
                    #path=nombre del recurso a llamar
                    path='alumnos/getAlumnos', http_method='GET',
                    #Puede que sea la forma en la que se llama desde la api:
                    #response = service.alumnos().listGreeting().execute()
                    name='alumnos.getAlumnos')

  def getAlumnos(self, unused_request):
    ...
    students_list = []
    ...
    # Logic to get the students list from another service with maybe more logic.
    ...
    alumnosItems.append(Alumno( id=idAlumno, nombre=nombreAlumno.decode('utf-8'), apellidos=apellidosAlumno.decode('utf-8') ) )
    return ListaAlumnos(alumnos=alumnosItems)
...
\end{lstlisting}

\noindent This is only an example, but the complete file has about 2000 lines in the first
approximation of the service.
Now take a look to the \emph{Flask} version, an example bellow,  having exactly
the same functionality of previous code segment.

\begin{lstlisting}[language=python,frame=none]
@api_gateway.route('/students', methods=['GET'])
def get_students():
    ...
    students_list = []
    ...
    # Logic to get the students list from another service with maybe more logic.
    ...
    return students_list
\end{lstlisting}

\noindent The reasons to select Flask insetead of it?
Is not necessary define the schema of the objects, this is sufficient reason
to dismiss their use, but sound enough nice to considere to another
applications in the future (as almos any technology researched).


\subsection{As a simple dispatcher}

This is an example of the behavior dispatching a simple request and answering
exactly the same response from the service.

\begin{lstlisting}[language=python,frame=none]
  @tdbms_segment_api.route('/entities/<string:kind>', methods=['POST'])
  def post_entity(kind):
      response = requests.post(url='http://' + str(modules.get_hostname(module='tdbms')) + '/entities/' + str(kind),
                               json=request.get_json())
      response.headers['Access-Control-Allow-Origin'] = "*"
      return make_response(response.content, response.status_code)
\end{lstlisting}


\begin{lstlisting}[language=python,frame=none]

  @tdbms_segment_api.route('/entities/<string:kind>', methods=['GET'])
  def get_entity(entity_id):
      ...
      response = requests.get(url='...', json=request.get_json())
      ...
      response['profile_image'] = requests.get(url='filesServicesUrl...')
      ...
      return make_response(response, 200)
\end{lstlisting}

\noindent And the next natural step will be built a library that works as a general
customer of any service in the system, to be used in any place where be needed
make a call to any service. This will to work loading dynamically at execution
time the api of each service to know what resources are available and turned
this as Python methods.


\begin{lstlisting}[language=python,frame=none]

  @tdbms_segment_api.route('/entities/<string:entity_id>', methods=['GET'])
  def get_entity(entity_id):
      ...
      response = service['teachingDBmS'].getEntity(entity_id)
      ...
      response['profile_image'] = services['filesStorage'].get(file_id)
      ...
      return make_response(response, 200)
\end{lstlisting}

\noindent With this new tool, the code of the whole project will be reduce 10\% at least.

\section{Teaching Data Base microService}

\subsubsection{Optional subjects, the ``class'' Table.}

In the domain of the problem can be exists optative subjects and is
needed search a way to implement this because has a specific details
that aren't like the rest.

\paragraph{The details.}

The studies plan forces in certain courses to select one or several
optional subjects. For example, if a student has enrollment in 2ºESO
(independently of the group, A, B...) the law and consequently the
studies plan force to the student to choose between some optional
subjects. So, maybe this subjects exists only in this optional case
as `` rare subjects'' but in other cases this are only normal subjects
but that in this course are offer like optional.

A simple example of this is French subject, it in some courses like
3ESO and 4ESO is obligatory but it in Bachiller (the upper level)
is optional because the students can be select if they want make the
final exam with this second language or select another like English
or Greek or Latin p.e.

To obtain this we decide develop a simple solution without change
the original database logic schema. So, how we have an entity that
save the classes and it have three attributes, course, word and level
mainly we going to add three more to this special cases, optative,
groupNumber and subgroupNumber. Like this special cases haven't word
param when they have value word don't have and when the item have
word (A, B, C...) then don't have this special attributes.

Maybe this don't be the best solution, but is a simple in the point
of develop.

Obviously like we can't have two autoincrement values in the same
table definition in MySQL we will need control this programatically,
but is something that we can assume to get our goal easily.

Problema:

La misma ventaja que nos ofrece UNIQUE para el caso de los eliminados
ahora nos muerde aquí. Mientras que allí beneficiaba porque esta clausula
no incluye a los item que tengan campos a null y permite que no de
conflicto en este caso si insertamos un grupo optativo obviamente
deberá tener el camo word a null y si es así podemos tener exactamente
dos grupos iguales sinque de conflicto.

course

1 <null> ESO 1 1 1 0

1 <null> ESO 1 1 1 0

Sin dar conflicto, lo que no puede ser.



está a null o no para le resto de los procedimientos.

Por esta razon decidimos usar el mismo campo word, con una nomenclatura
especial, ya que no se va a ser usada para los grupos generales que
indique que se trata de un grupo optativo y además especifique el
grupo y el subgrupo, en concreto:

OPT\_n\_m

donde n será el número del grupo (que habrá que incrementar prograticamente
a mano) y m el número del subgrupo que tambi\'{n}e habrá que incrementar
a mano en caso de que se creen más.

Esto siempre a falta de un mejor solucion que se adopte en iteraciones
posteriores.

Y en la api nosotros siempre preferimos explícito a implicito.

Requisitos:
\begin{enumerate}
\item Ya que todo se controla desde mysql pero no existe forma de hacerlo
automatico vamos a crear unos disparadores para la inserccion y la
eliminacion para mantener la consistencia de los datos. Cuando se
introduce un nuevo grupo optativo se debe comprobar que existe el
anterior tanto en grupo como en subgrupo para que no exista el subgrupo
4 sin el 3 o el dos. De la misma manera no se podrán eliminar un grupo
4 si el 2 aún existe.
\end{enumerate}

\paragraph{API definition}

@app.route('/entities/<string:kind>', methods={[}'POST'{]})

Para realizar la introduccion de un estudiante al sistema usamos

el re.

No podemos introducir un usuario que no tenga datos, al menos

deberá pasarse el dato nombre que es úniquo mínimo que debe

existir para el tipo alumno.

Cuando se pide una lista de elementos y la peticin se realiza correctamente
pero no se devuelve nada porque no existen items de ese recurso se
envía un 204 Success without content.

Importante: Muchos de los detalles de control de consistencia no pueden
relegarse a la UI, ya que aunque en ella no se deben permitir ciertas
acciones por la logica del sistema, la api no debe permitir su realizacin
aunque se haga de forma manual sin los conectores de la UI.

Por eso no solo la seguridad sino el control de la consistencia debe

estar presente de la mejor forma posible en todas las acciones que
contra esta un usuario puede hacer.

\paragraph{User histories}

El microservicio de Base de Datos acota el dominio de la gestin docente
dentro del sistema, quizs se le cambie el nombre a TmS teachingMicroService.

A través de este microservicio podemos realizar todas las acciones
relativas a.

Este microservicio se basa en una base de datos.

I like

\subsection{Deleted items strategy}

\section{Students Control microService}

\section{User Interface microService}




\subsection{Save processes flows}

The logical process of saving data is apparently easy but it hides some details
when we talk about update existing data. These and how to solve the problem that
it presents will define how the user will use our interface. And spite of this is
a design phase issue is in the develop moment when this appeared and is why it
is explained here.
\linebreak
\linebreak
\noindent Focused on the problem, we have an object load in the interface, as a complete
profile info of a student, and we want to update their data (change or add some
new data) and this is not trivial because will define the signature between the
API of service and the user interface.So, basically, we have found three ways to do this so we are going to analysis
each one to choose the best, even though there are combinations.
When the object is loaded in our interface we modify some field and push the save
button if all the form requirements are satisfied then the complete object is sent
without any else requirements.

\begin{enumerate}
\item \textbf{Common save button}

The object is sent always, independently if it really has suffered changes.
Save button is always available while the requirements of the form are satisfied.

\textbf{Disadvantages}:
The object is always sent, even if is not necessary because it has not suffered
changes, hence a lot of bytes will be sent without sense.

\textbf{Advantages}:
Is the simplest and faster implementation for the user interface and the server
do not need to check anything
only override the object updating their metadata.

\item \textbf{Automatically saved}

The object is sent each time as it suffered any change,
save button not exists. Their behavior is similar to an online text editor where
all changes are saved implicitly.

\textbf{Disadvantages}:
The calls needed are huge if the user to do a lot of changes. Is necessary another
parallel mechanism to maintain the possibility to do undo because if all changes
are saving at the moment in the server there are not any way to do a simply undo action.
A number of calls needed are huge.

\textbf{Advantages}:
Is really futurist because the user does not to need to push any button after of
update their data. All forms become more clear.

\item \textbf{Totally checked}

Is a more efficient way that the rest. In this case, when the object is loaded
in the interface a copy is done and
saved also, always, independently of the interaction of the user. With this, we
achieve to get a copy of the object without any modification when the user is updating some field.

So, each way that user change some field of the form the logic check if there are
any difference between both objects (original copy and the modified) and only
when this difference exists and the requirements of the form are satisfied the
save button will change to enable.
This way if the user cancels the update of the object or after of thousand
modifications leave the object exactly with the same data the send button will
not enable because there are not any to update in the server.

\textbf{Disadvantages}:
The computing requirements in the user device it will high because it will be
needed hundred of checks in a simple interaction.

\textbf{Advantages}:
The server is not involved until the object is really updated. It supposes a user
 experience that gives to interface more intelligent aspect.
\end{enumerate}

\noindent As is easy to imagine the option chosen is the third because the reason explained
above. Down below is shown a piece of code which we have developed the feature of object copy.
When we can to see how is used a variable to save the state of the object, after
the object is saved and watches is enabled with it to detect any possible change
that will modify the state of save button, all after of the student object has been returned.

\begin{lstlisting}[language=java,frame=none, title=studentProfile.js]
...
function loadData() {
    vm.student = StudentsService.get({id: vm.studentId}, function () {
        ...
        vm.studentOriginalCopy = angular.copy(vm.student);
        $scope.studentModelHasChanged = false;
        $scope.$watch('vm.student', function (newValue, oldValue) {
            $scope.studentModelHasChanged = !angular.equals(vm.student, vm.studentOriginalCopy);
            if ($scope.studentModelHasChanged)
                vm.updateButtonEnable = true;
            else
                vm.updateButtonEnable = false;
        }, true);
    }, function () {
        console.log('Student not found')
        ...
    });
    ...
}
...
\end{lstlisting}

\noindent So, this behaviour is replied in all places where this kind of interaction appears.

\includepdf[pages=-,pagecommand={},scale=0.94]{img/diagrams/addTeachingToTeacher_AD.pdf}

% \begin{figure}[H]
%   \includegraphics[scale=0.2]{img/snaps/teachers_list.png}
%   \centering
%   \caption{Teachers List view wireframed.}
% \end{figure}
%
% \begin{figure}[H]
%   \includegraphics[scale=0.2]{img/snaps/teacher_profile.png}
%   \centering
%   \caption{Teachers List view wireframed.}
% \end{figure}

\noindent Teachers list

\begin{figure}[H]
\centering
\begin{minipage}{.5\textwidth}
  \centering
  \includegraphics[scale=0.3]{img/snaps/teachers_list.png}
  \caption{A figure}
\end{minipage}%
\begin{minipage}{.5\textwidth}
  \centering
  \includegraphics[scale=0.3]{img/snaps/teachers_list.png}
  \caption{Another figure}
\end{minipage}
\end{figure}

\noindent Teacher profile

\begin{figure}[H]
\centering
\begin{minipage}{.5\textwidth}
  \centering
  \includegraphics[scale=0.3]{img/snaps/teacher_profile.png}
  \caption{A figure}
\end{minipage}%
\begin{minipage}{.5\textwidth}
  \centering
  \includegraphics[scale=0.3]{img/snaps/teacher_profile_2.png}
  \caption{Another figure}
\end{minipage}
\end{figure}

\noindent Updating a teacher profile (note that this is
the spanish version).

\begin{figure}[H]
\centering
\begin{minipage}{.5\textwidth}
  \centering
  \includegraphics[scale=0.3]{img/snaps/teacher_profile_update.png}
  \caption{A figure}
\end{minipage}%
\begin{minipage}{.5\textwidth}
  \centering
  \includegraphics[scale=0.3]{img/snaps/teacher_profile_update2.png}
  \caption{Another figure}
\end{minipage}
\end{figure}

\noindent Students lists and graphics.

\begin{figure}[H]
\centering
\begin{minipage}{.5\textwidth}
  \centering
  \includegraphics[scale=0.3]{img/snaps/teacher_profile_students.png}
  \caption{A figure}
\end{minipage}%
\begin{minipage}{.5\textwidth}
  \centering
  \includegraphics[scale=0.3]{img/snaps/teacher_profile_graphics.png}
  \caption{Another figure}
\end{minipage}
\end{figure}

\section{Analysis microService}

PONER EJEMPLOS DE COMO LAS ESPECIFICACIONES DE RUTAS EN LA PARTE DE DISEÑO
SE CORRESPONDEN CON LA IMPLEMENTAIÓN EN FLASK
