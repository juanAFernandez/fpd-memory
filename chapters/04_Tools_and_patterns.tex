\chapter{Tools, patterns and firsts decisions}

Nowadays we can build almost any kind of software with a lot of differents
language from which to choose, but as if that were not enough there
are also a lot of architectures that are possible to follow, and beside
of this it will be necessary choose where to deploy our software, or how
doc it, or how to manage our work team, etc. A lot of options that it
will be necessary to choose and that can make the difference between the
success or failure of our project. So, what about of these decisions
in this project? Take a look at it.

\section{Architecture}

Almost any kind software or tool can be built of differents ways, independently
of this behavior or the goals that it must achieve. Some ways are more specific
to achieve some specific behavior and another.

\subsection{Kind of software}
Based on informal requirements was thought that the best option must be a
\textbf{web app} for a lot of reasons. First of all because is the simplest
way to offer an app that can run on almost any device (with a \textit{
simple\footnote{Actually a browser is one of the more complex pieces of
software, but here is labeled as simple because is a software that the
majority of devices like smartphones, tablet, etc, have by default and for
the most nontechnical people isn't a complex software, nothing could be
further from the truth.}} browser).

\subsection{Why microservices?}

When we think about an app, of any type, the most of the time we think of a software
running on a single machine, with more or less hardware available, where all
related to the software are inside of the same machine.
\intro
Now, we hearing all the time about microservices, which are in synthesis the
opposite of the traditional monolithic classical approach in software architecture,
and seem like if your design is not based in this mean you are outdated or your design is directly wrong.
Well, this is not totally true but neither false. The goal of this section is
not describe the advantages and drawbacks of this approach, but yes justify why
is their choice.
\intro
Microservices are actually a variant of SOA\footnote{Service-oriented architecture (SOA)
was first described by Gartner in 1996 and is a style of software design where
services are provided to the other components by application components, through
a communication protocol over a network},that simplifying a lot is the architecture
of a software system composed of few
different systems, working together to do the service which the software was design.
Dr. Peter Rodgers introduced the term "Micro-Web-Services" during a presentation
at Cloud Computing Expo in 2005 and as we can see in the following graphic (extracted
using sotagtrends.com) community interest about this emerging technology
has never stopped growing,

\begin{figure}[h]
  \includegraphics[scale=0.5]{img/graphics/microservices_trend.png}
  \centering
  \caption{Search trend of "microservice" tag in Stack Overflow.}
\end{figure}

James Lewis\footnote{Principal Consultant at ThoughtWorks and
member of the Technology Advisory Board} and Martin Fowler\footnote{British software developer,
author and international public speaker on software development, specializing in
object-oriented analysis and design, UML, patterns, and agile software development
methodologies, including extreme programming.} are the major defensor of this
arquitecture and this brings us to the canonical definition of microservices:


\begin{minipage}{0.9\linewidth}
        \vspace{5pt}%margen superior de minipage
        {\small
        \textit{"In short, the microservice architectural style is an approach to developing a
        single application as a suite of small services, each running in its own process
        and communicating with lightweight mechanisms, often an HTTP resource API.
        These services are built around business capabilities and independently deployable
        by fully automated deployment machinery. There is a bare minimum of centralised management
        of these services, which may be written in different programming languages and
        use different data storage technologies."}
        }
        \begin{flushright}
            (Lewis/Fowler)
        \end{flushright}
        \vspace{5pt}%margen inferior de la minipage
    \end{minipage}


\noindent Language agnostic, scalable, size ajustables, independently, our system design
splited in litle pieces with this boundaries do this architecture perfect for
our goals, and this is why did this choice.

\subsection{Why polyglot database?}

With the mircroservices approach the system will be some very differents
database to do some diferents things. So, in general we can see our
backend like a black box when the data persists in a polyglot database.
That means that the data is save in differents ways, using diferetns
formats and diferents driver to manage this. There are a group of
data that it need saved with certain relations, due to its nature,
so a relational database seem perfect to do this, but this kind of
databases (like MySQL) can been slowly or too heavy for other tasks
of kind of processes, like data analysis.

\section{Code strategy}


Entire project will be stored in a single repository, using git as control
versions system.

\subsection{Version Control System}

Git\footnote{Git is a free and open source, distributed version control system
created by Linus Torvalds} is the perfect tool to achieve maintain a more o
less easy workflow between developers minimizing the risks, and the huge list of
plugins to different software and IDEs do this perfect to work, dismissing another
as Subversion, Mercurial, Fossil, or Bazzar.
\intro
The reason to use a single repository is that the mechanism to maintain the
consistency between projects is not easily enough to do the develop agile.
In spite of, git offers us a lot of tools that can improve enormously our
workflow, as the Git Submodules\footnote{Submodules allow you to keep a
Git repository as a subdirectory of another Git repository.
This lets you clone another repository into your project and keep your commits
separate.}, or Git Hooks\footnote{Git has a way to fire off custom scripts when
certain important actions occur. There are two groups of these hooks:
client-side and server-side. Client-side hooks are triggered by operations such
as committing and merging, while server-side hooks run on network operations
such as receiving pushed commits} between another.

\subsection{Hub}

For another hand we have to do another choice, select the remote hub of our repositories
or Git repository hosting service. The most popular for some years has been GitHub,
are a web-based Git or version control repository and Internet hosting service.
It offers all of the distributed version control and source code management (SCM)
the functionality of Git as well as adding its own features,
launched at 2008
but there are other options as Bitbucket\footnote{web-based hosting service
that is owned by Atlassian, used for source code and development projects that
use either Mercurial (since launch) or Git
launched at 2008},
SourceForge\footnote{another system launched at 1999} and especially
GitLab\footnote{launched at 2011, was written by Dmitriy Zaporozhets and Valery
Sizov from Ukraine. The code is written in Ruby. Later, some parts have been
rewritten in Go.} (with a big grow ultimately).


\subsection{Develop workflow}

Another of the main problem easy to find in the develop when are involved
few people is the organization of the contribution of a repo, but if the project
is composed of few subprojects, this can be worst. For this reason was choose the
strategy to have an only repo with subfolders.
So, took this decision remains to be defined how will work the repo in the
different phases or the develop.
\intro
To do this, we will follow the standard, using branches to develop features and
maintain a good state of the versioning of project.
\intro
So, as we can see in the picture, we are going to use \textit{develop} branch
to do the develop, where we are going to work in minimal changes and will do
a fork when we want to do some representative change in the code, as new functionality or
some big correction.


\begin{figure}[H]
  \includegraphics[scale=0.25]{img/git/feature_branches.png}
  \centering
  \caption{Workflow with features branches}
\end{figure}

\noindent For other hand when the all code to launch a final release we will fork the
repo in a \textit{release} branch to while the team continue with the develop
in the mainly develop branch ot in a issue form of this another part of the
team (or itself, is the same), will prepare the code to production, taking a look
because there are some bug, fix it and when all are testend an correctly working
put the project in exactly this verison in master (doing a merte) and this is
when officially a new version of program will be launched.

\begin{figure}[H]
  \includegraphics[scale=0.5]{img/git/release_branches.png}
  \centering
  \caption{Workflow with releases branches.}
\end{figure}

\noindent There are some variations of this standard workflow but for this kind of project
is perfect.

\section{Languages and frameworks}

As has been said before, almost any kind of software can be builded at infinite
ways, and this start with what language can do this. This means it could be
builded with Java, Ptyhon, C++, Go, Ruby, JavaScript, PHP,  (only talking about
backend) and with another list to frontend, in this deep list of possibilities,
we choose the most flexible and powerfull of all of them, Python and JavaScript.
\intro
Python because is one of hte most simplest and powerfull languages nowadays, and
JavaScript because is all an standard in the industry. Obviously the choise is
based also in the fact of both languages are really supported of community, have a
good learning curve and a lot of projects and systems are based in them.
\intro
JavaScript has been selected becuase Angular is writte in it. AngularJS is the
most powerfull framework nowdays to buil fas, clean and powerfull web apps.


\subsection{Communications}

To exchange data between service we need to select a idiom which the services
will talk, which them will exhange information. That's mean mainly select how
will transform the objects anda data structures that servicres manage to be able
sender across the net.

\subsubsection{Data serialization format}

We have some data serialization solutions, some very popular and another most
specific of very focused domains. So, the most commons are Json, Yaml,
Bson and ultimately MessagePack. Each have their owns benefits and drawbacks but
the selection has been easy, JSON.
\intro
\textbf{XML}
\intro
Extensible Markup Language, is a markup language, defined v1.0 by
W3C\footnote{World Wide Web Consortium, founder by Tim Berners-Lee at 1994 at MIT
(Massachusetts Institute of Technology) the consortium is made up of member
organizations which maintain full-time staff for the purpose of working together
in the development of standards for the World Wide Web.} This is an example:

\begin{lstlisting}[language=xml,frame=none,numbers=none]
  <exam>
    <result>5.8</result>
    <type>Partial</type>
    <subject>Science</subject>
    <date>17-06-2018</date>
  </exam>
\end{lstlisting}


\noindent \textbf{JSON}
\intro
JavaScript Object Notation, is an open-standard file format that uses
human-readable text to transmit data objects consisting of attribute–value pairs
Douglas Crockford originally specified the JSON format in the early 2000s;
two competing standards, RFC\footnote{ Request for Comments, is a type of
publication from the Internet Engineering Task Force (IETF) and the Internet
Society (ISOC), the principal technical development and standards-setting bodies
for the Internet. were invented by Steve Crocker in 1969 to help record
unofficial notes on the development of ARPANET} 7159 and ECMA\footnote{Ecma is a
standards organization for information and communication systems founded in 1961
to standardize computer systems in Europe.}-404, defined it in 2013.
The ECMA standard describes only the allowed syntax, whereas the RFC covers some
 security and interoperability considerations.
\intro
In spite of A restricted profile of JSON, known as I-JSON (short for "Internet JSON"),
defined in RFC 7493, is not as popular as original. This is an example:

\begin{lstlisting}[frame=none,numbers=none]
  {
    "title": "The Picture of Dorian Gray",
    "author": "Oscar Wilde",
    "date": "July 1890"
  }
\end{lstlisting}


\noindent \textbf{YAML}
\intro
Yet Another Markup Language is commonly used for configuration files, but could
be used in transmision also, YAML 1.2 is a superset of JSON, whitch Latest release1.2
(Third Edition) was published at (1 October 2009; 7 years ago), YAML was first
proposed by Clark Evans in 2001.

\begin{lstlisting}[frame=none,numbers=none]
  ---
  invoice: 34843
  date   : 2001-01-23
\end{lstlisting}


\noindent \textbf{BSON}
\intro
Binary JSON, is a computer data interchange format used mainly as a data storage
and network transfer format in the MongoDB database.
MongoDB represents JSON documents in binary-encoded format called BSON behind
the scenes. BSON extends the JSON model to provide additional data types,
ordered fields, and to be efficient for encoding and decoding within different languages.
\intro
\noindent \textbf{MessagePack}
\intro
Byte array is an efficient binary serialization format, like JSON but faster and smaller.

\begin{lstlisting}[frame=none,numbers=none]
{"compact": true, "schema": 0}
27Bytes

82 A7 compact C3 A6 schema 00
18bytes
\end{lstlisting}
\noindent And another aprox as zerorp, It builds on top of ZeroMQ and MessagePack

\subsubsection{Protocol}

Some introdfdsf dafdaf adsf ds fdsf da fadf dsdsfdsfdsf aad adsf ads adsf adsf afs
\intro
\noindent \textbf{APIRest}
\intro
Representational state transfer (REST) or RESTful Web services are one way of
providing interoperability between computer systems on the Internet. REST-compliant
Web services allow requesting systems to access and manipulate textual representations
of Web resources using a uniform and predefined set of stateless operations. Other
forms of Web service exist, which expose their own arbitrary sets of operations such as
WSDL and SOAP
\intro
\noindent \textbf{RPC}
\intro
In distributed computing, a remote procedure call (RPC) is when a computer program
causes a procedure (subroutine) to execute in another address space (commonly on
çanother computer on a shared network), which is coded as if it were a normal (local)
procedure call, without the programmer explicitly coding the details for the remote interaction.


\noindent \textbf{gRPC}
\intro
gRPC is an open source remote procedure call (RPC) system initially developed at Google. It uses HTTP/2 for transport, Protocol Buffers as the interface description language, and provides features such as authentication, bidirectional streaming and flow control, blocking or nonblocking bindings, and cancellation and timeouts. It generates cross-platform client and server bindings for many languages.

We will talk a bit more about this in the Develop chapter.


\subsection{Testing}

The tool that we will use for any testing involved in the backend of the project
will be PyTest. Launched at 2004 by Holger Krenel is one of the most complete
suites for testing over Python.
Is really easy to use and allow do some things that is very difficult with another
frameworks as built-in \textit{unittest} python package, somethings as the use of
fixtures of or the amount of plugint that it has.
\intro
Focused on the problem it will be used to make all unittest of the core of services,
libraries and auxiliary programs and testing the entire functionality of the service
when it has a role of black box.


\subsection{Documentation}

As is saided the tool Sphinx is used to build the doc of the service.
That basically inspect the code files mixing this with all the files
that we write (pure doc) to show this as a web based documetnation
(easy to read and understand).

\section{Storage}

With the storage occurs the same, there is a huge list of options to
choose. The fast answer in the most of the cases is: Why do not use a relational
database, as MySQL for example? If we are thinking in a little system maybe a
good choice, always that we data model is adjusted to this kind of system. But
every people knows the deficits and the complexity of developing a system
with this database. Mainly the performance when we are talking about million
of arrays of objects stored. So, as we are talking before, our goal in this project
is work with a lot of data, which apart are pure relational and another can be
processed by another way.
\intro
If you have a logical data model like as this project have is not easy choice one
of all database engine to model this. SQL systems is very powerful for some
things but not for another (or not simply) and object oriented databases is
really powerful and simple to develop with it when the data haven't a lot of
relations (although can be modeled also).
so, why choice one between them? Why do not both? This is the approach selected,
build a system with a Polyglot Database, that means: much better select the better
engine for any kind of data instead of trying to use the same for all.
\intro
And this approach joined with microservices architecture give as the first
conceptually of our system, when each microservice of domain work with their
data, using for this their own database engine, their own kind of data, and
until their own language if is precise.
\intro
So, focused on this project, we are going to use mainly two systems, SQL
relational database system in a service (we will talk more about it after),
and a NoSQL service, in this case, Google Datastore, a fast and lightweight
engine to mange an object oriented database.
\intro
At the moment of to write this chapter had been evaluated MongoDB as the best
choice, but the facilities of the platform selected (detailed after) was made
that the G. Datastore was selected finally. Beside of this if the project are
rebuild now have not doubt, Mongo will be selected after the experience ( the
reasons will be explained in more detail in the conclusion of the work).

\subsection{Versions}

Talking about Python has selected 2.7 version not for the best reasons.
And is because the sandbox of Google App Engine do not allow another upper versions.
\intro
About Angular the selectioni Angular1.x. Seem a bit old, the new version 2.x
is really popular now and the community is talking about the next two levels
uppper version AngularJS 4.x.
\intro
The selection of the lower version is simple, a lot of interesting material
about this, the most of problems that we can find already solved or with a lot
of help and the maintenance.

\subsection{Frameworks}

\subsubsection{Flask}

At the moment to write this work has been used it but exists another
that are been used in another project that have more potentia because have
better performance, as Falcon (betweern anothers).


\section{Platforms}

Today, with the explosion of the cloud is needed a lot of infrastructure that support
all that all companies that want to migrate thier bunisses model to cloud need. the possiblite
come from of the companies that offer solutions to do this, and another that after
was introduced in the same bunisses.
So, actually there are a lot of companies that offer some solutions to sofware (not necessary)
companies to work with cloud to build their solutions and cloud computing easily.
\intro
There are a lot but only few are the big in this busineess. And alwyas respond to some
model IaaS (Infrastructure as a Service)
AWS, G Computer Engine, Rackspace Cloud, vCloud PaaS (Platform as a Service)
Google App Engine, Heroku, SaaS (Sofware as a Service) Google Docs, Salesforce, Dropbox, Gmail
\intro
\textbf{Amazon Web Services}
\intro
Launched at 2006 The most popular include Amazon Elastic
Compute Cloud (EC2) and Amazon Simple Storage Service (S3)
\intro
\textbf{Microsoft Azure}
\intro
is a cloud computing service created by Microsoft for building, testing, deploying,
and managing applications and services through a global network of Microsoft-managed
data centers. It provides software as a service (SAAS), platform as a service and
infrastructure as a service and supports many different programming languages,
tools and frameworks, including both Microsoft-specific and third-party software and systems.
Azure was announced in October 2008 and released on February 1, 2010 as Windows
 Azure, before being renamed to Microsoft Azure on March 25, 2014.
\intro
\textbf{IBM BlueMix}
\intro
is another platform.
\intro
\textbf{Heroku}
\intro
Is a cloud platform as a service (PaaS), founded by James Lindenbaum in 2007 at
 San Francisco, California, supporting several programming languages that is used
 as a web application deployment model.
 Heroku is said to be a polyglot platform as it lets the developer build,
 run and scale applications in a similar manner across all the languages.
 Heroku was acquired by Salesforce.com in 2010 for 212 million of dollards.
\intro
\textbf{CloudFoundry}
\intro Is an open source, multi cloud application platform as a service
(PaaS) governed by the Cloud Foundry Foundation, a 501(c)(6) organization.
The software was originally developed by VMware and then transferred to Pivotal Software,
a joint venture by EMC, VMware and General Electric launched at 2011
\intro
\textbf{Google Cloud Platform}
\intro Is a suite of cloud computing services
that runs on the same infrastructure that Google uses internally for its end-user
products, such as Google Search and YouTube.Alongside a set of management tools,
it provides, a series of modular cloud services including computing, data storage,
data analytics and machine learning.
\intro
And is this which was selected.

\section{Documentation}

To do the documentation of the code as in the rest of sections, we have a few options
to choose. So, as the backend is fully developed in Python we are going to use some python
especially documentation framework and in this case Sphinx\footnote{Sphinx is a
simple and powerful Python-based documentation generator.} has been selected.
\intro
For another hand, to the front, especially talking about Javascript with AngularJS
we do not have the need to a doc because the docstrings have been sufficient,
and neither have been found any really good tool as Sphinx for them.

\section{License}

In this project, we are going to use GNU General Public License v3 (GPL-3)
to license our code.
There are a lot of licenses available, but it meets all our needs,
this is a little summary of this:

\begin{table}[H]
  \centering
  \begin{tabular}{ l | c | r }
    Can &  Cannot & Must \\
    \hline
    Commercial use    & Sublicense    &  Include original \\
    Modify            & Hold liable   &  State changes \\
    Distribute        &               &  Disclose source \\
    Place warranty    &               &  Include License \\
    Use patent claims &               &  Include Copyright \\
                      &               &  Include Install Instruction \\
  \end{tabular}
  \caption{License characteristics summary.}
\end{table}
