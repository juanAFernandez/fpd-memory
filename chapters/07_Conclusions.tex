\chapter{Conclusions}

Only a few months might seem a little bit but  is enough to obtain some
conclusion about techonologíes, patterns and ways to develop.
This work can not finish without a reflexion about what have been the mainly
drawbacks and locks in the develop, design and evolution of the entire project.


\subsection{Contra la planificaión}

Explicar como se ha aumentaod el tiempo de desarrollo y como los spikes
fueron mucho más necesario y el cambiar de tecnologías varias veces y los
problemas planteados. Un par de gráficas, etc...

\subsection{Technologies and frameworks}

Of one side, we have the difficulties with the technologie choised. In this case
 is can saided that use AngularJS and Python has been literally perfect, beyond
 that the typical novice errors with the languages and their learning curves.
 So, in generall without dude they will be choised as technologies again.

 About the platforms or technologies the point of view changes. If you are
 really expert developer use platforms like Google App Engine can be really
 interesting, because you are evaluated the rest of the options, but when
 you don't have any practice, in my view, is not a good option. Moreover when
 the learning curve is so soft.

 As in many new technologies is easy do the first steps.


 Por una parte hay que tener mucho cuidad al elegir una tecnología emergente,
 pero po otro lado, no nos vamos a basar solo en los estándares por miedo
a perder el tiempo en innovación, si no todo el mundo seguiría programando en
C++, JAva o PHP.
Los pitch de aprendizaje están más que jutificados, y quizás son mucho más necesarios
y debería de darsele mucho más valor de el que tienen.

Docker en lugar de Google App Engine, aunque si la plataforma para desplegar y los motores de base SQL para datos,
AngularJS 1.x una buena elección pese a la antiguedad,  Mongo en lugar de DataStore debido a la
sencillez y la curva de aprendizare tan buena.


\subsection{The develop process}

Hablamos de organización del trabajo en equipo, de las contribuciones de terceros,
de como llevar equipos de trabajo, y de como captar, evaluar y unir a gente de
distintas disciplinas.

El diseño y el tipo de iteraciones. Con mucha experiencia myuy bien
sin experiencia, un putooooo caos.



\subsection{Opportunities}

Take part of the some kind of software contest is the better decision that any
software student can take. Visibility, networking, new friendship are only some
benefits that can achieve.
Thanks to enroll this project of the contests that the Open Source Department of
 the University of Granada with JJ Mereo as principal organizes was offer a job
 in a related software company with the techonologies and patterns ussed.
So, if any people think that the participation, the contribution and involvment
is not usseful, is absolutely wrong. In all of cases, this attitude front the
studiens only return benefits.

\subsection{Future of the project}

At the beginning of the develop, the idea behind of this was put in production
the result in a few months in beta mode in a school center of Granada, but now,
the jobs oportunities referred above have been done that the project go to the
another plane, less important, because the ideas and the philosofy are been
developed just now with another really good engineers in a company, building a
privative software, architecture and new related tools.

So, independiently the license of the code has not changed, and the develop can
 go ahead with any developer or group of them that want.

For another hadn the continuous evolution of the technologies do this issue a
bit difficult and actually it is another learned lesson about the innovate
software using three party techcnologies, we never will have safety that the
technology never will change. If you are working with C++ or even Java with your
own infrastructure the changes are minimal thorugh the months, with third party
technologies and support you need be at day with all changes and update almost
all your software each year. So, is difficult in this cases that the continuity
of the project will be ensure, at least without the original designer inside the
new developers teams. But, anyway, is only a point of view, with the software
nothing can be assumed.

Independly, the code is open, to learn , to review, and maybe to help someone,
so for this part, we are happy.

\subsection{Open Source}

Another conclusions getted in the develop are related with open source and the
viability to survive on this. Many time in the college is easy to hear that the
open source is a good way to start and is true, but not if you want to work of
this. Work in open source projects is really interesting, for the community,
for the workflow, for build some usseful to the community and by a huge list of
advantajes. But this is possible only when, for one hand you ar working in a
compnay and some of projects of this is decide be open, independiently of the
reason, community, better visibility, etc, or when you are student and have
the oportunity to free amounts of code, as this case. But for other hand,
think to build a company, more o less big based on a open source souliton
is very very rare and complex, mainly to younger and inexpert  software developers.

Obviously in the most of cases, always there are some exceptions that are
wonderfull examples that project with a amazin grow, and a really amount of
code that any developer must have would be open, always open, because there
are any developer that can be learn alone, without the community (in any of their
forms) and be in the obligation to contribute, to give back the favour.
