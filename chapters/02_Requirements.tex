\chapter{Requirements}

In this chapter, wer are going to describe which are the requirement that our system has.
It will be used User Stories\footnote{A user story is an informal, natural language
description of some features of a software system. See bibliography to know more.}
methodology which it will be talked a bit more in the next chapter.
This tool will be useful to identify the requirements of the system by a
human readable way., removing all possible layer between design and traditional
requisites specification process.
\intro
User stories can be very minimal (unit) and very general. That mean that we
can do a user story that is a simple requirement or a very general issue.
If the criteria to follow is to do very minimal users stories, we could be
hundred of them, so in this case, is better have an upper abstraction level
to have fewer user stories but with more content, at least when is the
first time that the team uses this methodology.
\section{Minimal user requirements}

The managers of the center can't have a precise idea about which must be the
functions of the software. They have a clear idea about what kind of process in
their daily work there are the heaviest and consume more paper, that is which
they want to digitalize, but they have not thought about navbars, icons, colors palettes,
user interfaces design or something like this.  They have abstract ideas that will
need dimensioning and shaping but which in synthesis offer the minimum system requirements.

\subsection{General needs}

How is talking about software, it will need run in somewhere, and they what to
be cheap and easy to run. And at first, there are not any pre-requirements. Maybe
they prefer that this can run in devices like smartphones, tablets, laptops,
but they do not have a good technical knowledge and for them is indifferent that
this will be a native app, a web app or some strange artifact.
\intro
For other hand exists an essential requirement, must be run on any device for
the same person, independently of the machine where it be (sometimes in laptop
and sometimes in a smartphone, for example).
\intro
Please see user stories GN-1 to GN-3 on appendix A.


\subsection{Academic Management System}

As base of requirements process have been used user stories.
As we can imagine, a center have a lot of differents kind of relationships,
and as obligatory part of the system it must be offer a simple way to do this.
Teachers imparts subjects to group of students (called classes), students are
enrolled in subjects, and so on. These cases of use and all it can see below will
be revised as \textit{User Stories} in detail later.
\intro
Please see user stories AMS-1 to AMS-7 on appendix A.

\subsubsection{Attendance Controls}

One of most important process to digitally, the amount of data that is saved on
paper and also impossible to manage before and much less do it an analysis.
They need a simple way, a simple interface to do this that allow this save time
(doing it, analysis and management it), paper and effort.
\intro
Please see user stories AC-1 on appendix A.

\subsection{Class Controls}

A part of manage of classrooms they must offers a say to follow the students evolution in class, behaviour, positives and negativss and another lot of things like paper save, time and effort as section above.
\intro
Please see user stories CC-1 on appendix A.

\subsection{Marks}

As another of proccess that more paper consumes is the marks management. They want a system that simplity the proccess of insertion, analysis and management. Without any specially idea in mind are open to any good user interface that simplify this.
\intro
Please see user stories MKs-1 on appendix A.

\subsection{Disciplinary Notes}
\

It also must provide a management of this kind of notes, in which a bad behaviour of a studnet is saved, managed and fully reported to the specific users inside of app, as tutors, pedagogue, etc, for example.

Please see user stories DN-1 to DN-2 on appendix A.



\bigskip
\subsection{Simple and advance reports}
\bigskip

Another part interesting for they is improve the reports that they obtains from their data. The on paper support do this almost imposible to big scale, an a important feature must be do this posible. Advanced reports about a lot of kind of items, like students, state of subjects, marks, etc, in seconds with some cliks.
This will improve the take of decisions and will do better meetings with better decisions based on good and crontastable data.


An user stoire by subsystem: repot by marks, report by attendance control..
reports by.., etc ,etc etc....

Please see user stories SAR-1 on appendix A.




\bigskip
\subsection{Autonomic Official System Connection}
\bigskip

The national system of education in which this center is framed have a diferents informatic reginal system. In this case, in Andalucia, where the certer is it called SENECA (other in other places of the country). All public and semipublic educaitonal centers need save data in this systems necessarily, and this haven't a simple public API where connet us. They must be done dirty way, making a maxed mode way to simplify the download and download of data that minify the interaction with the official system.


Please see user stories SNC-1 on appendix A.

\section{System requirements}

about system, need to be Modularity, Comunications and something like this. 
Please see user stories SNC-1 on appendix A.
